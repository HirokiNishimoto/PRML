%\documentclass[twocolumn,~ 11pt,~a4paper]{jreport}
\documentclass[11pt,a4paper]{jreport}
\usepackage{amsmath,amssymb}
\usepackage{newtxtext,newtxmath}
\usepackage{ascmac}
\usepackage[dvipdfmx]{graphicx}
\usepackage{listings}
\usepackage[top=30truemm,bottom=30truemm,left=20truemm,right=20truemm]{geometry}
\begin{document}
% \setlength {\columnsep}{2zw}
%\twocolumn[
\title{PRML輪読会~資料}
\author{西本 洋紀 }
\maketitle
\chapter*{第7章~~~疎な解を持つカーネルマシン}
% ]
\subsection*{導入}
全ての点についてカーネルを計算するのは大変である.一部の点についてカーネル計算できたら便利である. 本節ではそのような方法として,
\begin{itemize}
  \item \textbf{SVM}
  \item \textbf{RVM}
\end{itemize}
を扱っていく. \\
\section*{7.1 最大マージン分類機}
SVMは
\section*{7.2 関連ベクトルマシン}

\end{document}

