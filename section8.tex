\documentclass[11pt,a4paper]{jsarticle}
\usepackage{amsmath,amssymb}
\usepackage{newtxtext,newtxmath}
\usepackage[dvipdfmx]{graphicx}
\usepackage{listings}
\lstset{%mactex
 language={C++},
 % backgroundcolor={\color[gray]{.95}},%
 tabsize=2, % tab space width
 showstringspaces=false, % don't mark spaces in strings
 basicstyle={\ttfamily},%
 %identifierstyle={\small},%
 commentstyle={\itshape},%
 keywordstyle={\bfseries},%
 %ndkeywordstyle={\small},%
 stringstyle={\ttfamily},
 %frame={tb},
 breaklines=true,
 columns=[l]{fullflexible},%
 % numbers=left,%
 % numberstyle={\small},%
 xrightmargin=0zw,%
 %xleftmargin=3zw,%
 stepnumber=1,
 numbersep=1zw,%
 lineskip=-0.5ex%
}
\lstset{%mactex
 language={R},
 % backgroundcolor={\color[gray]{.95}},%
 tabsize=2, % tab space width
 showstringspaces=false, % don't mark spaces in strings
 basicstyle={\ttfamily},%
 %identifierstyle={\small},%
 commentstyle={\itshape},%
 keywordstyle={\bfseries},%
 %ndkeywordstyle={\small},%
 stringstyle={\ttfamily},
 %frame={tb},
 breaklines=true,
 columns=[l]{fullflexible},%
 % numbers=left,%
 % numberstyle={\small},%
 xrightmargin=0zw,%
 %xleftmargin=3zw,%
 stepnumber=1,
 numbersep=1zw,%
 lineskip=-0.5ex%
}
\lstset{%mactex
 language={Python},
 % backgroundcolor={\color[gray]{.95}},%
 tabsize=2, % tab space width
 showstringspaces=false, % don't mark spaces in strings
 basicstyle={\ttfamily},%
 %identifierstyle={\small},%
 commentstyle={\itshape},%
 keywordstyle={\bfseries},%
 %ndkeywordstyle={\small},%
 stringstyle={\ttfamily},
 %frame={tb},
 breaklines=true,
 columns=[l]{fullflexible},%
 % numbers=left,%
 % numberstyle={\small},%
 xrightmargin=0zw,%
 %xleftmargin=3zw,%
 stepnumber=1,
 numbersep=1zw,%1
 lineskip=-0.5ex%
}
\begin{document}

\title{PRML輪読会 第4回 資料 \\
        グラフィカルモデル}
\author{氏 名: 西本 洋紀 \\
        教養学部学際科学科B群 2年
        }
\date{\today}
\maketitle
%-----------------------------------------------------------------------------------------------------
\section*{0 ~ 全体像}
「確率的グラフィカルモデル」とは...\\
確率分布の図式的な表現.\\
グラフィカルモデルは以下のような便利な特徴を持っている.
\begin{enumerate}
  \item 確率モデルの構造を視覚化する簡単な方法を提供し、新しいモデルの設計方法を決めるのに役立つ.
  \item グラフの構造を調べることにより、条件付き独立性などのモデルの性質に関する知見が得られる.
  \item モデルの推論や学習に伴う複雑な計算を、数学的な表現を暗に伴うグラフ上の操作として表現できる.
\end{enumerate}
\begin{itemize}
  \item 有向グラフベイジアンネットワーク
  \begin{itemize}
    \item 2 条件付き独立性
  \end{itemize}
  \item 3 マルコフ確率場
  \\
  \item 4 グラフィカルモデルにおける推論
\end{itemize}
\section{ベイジアンネットワーク}
\begin{itemize}
  \item グラフのリンクが特定の方向性を持ち矢印で描かれる
  \item 全結合 リンクが存在しないことを持ってグラフで分布の情報を表現。
  \item ここでは、有向閉路を持たないグラフ (有向非循環グラフ)を考えている。
\end{itemize}
\subsection{例:多項式曲線フィッティング}
\subsection{生成モデル}
\subsection{離散変数}
\subsection{線形ガウスモデル}
\section{条件付き独立性}
\section{マルコフ確率場}
リンクが方向性を持たない。
\section{グラフィカルモデルにおける推論}
\end{document}
