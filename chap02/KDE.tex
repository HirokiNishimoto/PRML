\documentclass[11pt,a4paper]{jsarticle}
\usepackage{amsmath,amssymb}
\usepackage{newtxtext,newtxmath}
\usepackage[dvipdfmx]{graphicx}
\usepackage{listings}
\lstset{%mactex
 language={C++},
 % backgroundcolor={\color[gray]{.95}},%
 tabsize=2, % tab space width
 showstringspaces=false, % don't mark spaces in strings
 basicstyle={\ttfamily},%
 %identifierstyle={\small},%
 commentstyle={\itshape},%
 keywordstyle={\bfseries},%
 %ndkeywordstyle={\small},%
 stringstyle={\ttfamily},
 %frame={tb},
 breaklines=true,
 columns=[l]{fullflexible},%
 % numbers=left,%
 % numberstyle={\small},%
 xrightmargin=0zw,%
 %xleftmargin=3zw,%
 stepnumber=1,
 numbersep=1zw,%
 lineskip=-0.5ex%
}
\lstset{%mactex
 language={R},
 % backgroundcolor={\color[gray]{.95}},%
 tabsize=2, % tab space width
 showstringspaces=false, % don't mark spaces in strings
 basicstyle={\ttfamily},%
 %identifierstyle={\small},%
 commentstyle={\itshape},%
 keywordstyle={\bfseries},%
 %ndkeywordstyle={\small},%
 stringstyle={\ttfamily},
 %frame={tb},
 breaklines=true,
 columns=[l]{fullflexible},%
 % numbers=left,%
 % numberstyle={\small},%
 xrightmargin=0zw,%
 %xleftmargin=3zw,%
 stepnumber=1,
 numbersep=1zw,%
 lineskip=-0.5ex%
}
\lstset{%mactex
 language={Python},
 % backgroundcolor={\color[gray]{.95}},%
 tabsize=2, % tab space width
 showstringspaces=false, % don't mark spaces in strings
 basicstyle={\ttfamily},%
 %identifierstyle={\small},%
 commentstyle={\itshape},%
 keywordstyle={\bfseries},%
 %ndkeywordstyle={\small},%
 stringstyle={\ttfamily},
 %frame={tb},
 breaklines=true,
 columns=[l]{fullflexible},%
 % numbers=left,%
 % numberstyle={\small},%
 xrightmargin=0zw,%
 %xleftmargin=3zw,%
 stepnumber=1,
 numbersep=1zw,%1
 lineskip=-0.5ex%
}
\begin{document}

\title{カーネル密度推定法}
\author{氏 名: 西本 洋紀 }
\date{\today}
\maketitle
%-----------------------------------------------------------------------------------------------------
\section{目的}
ある分布$p(\mathbf{x})$に従って、$N$個の観測値が得られたとする。この時、N個の観測値から、分布$p(\mathbf{x})$を推定したい。
\section{方法}
$\mathbf{x}$を含むある小さな領域$\mathcal{R}$を考える。この領域に割り当てられた確率$P$は
\begin{equation}
  P = \int_\mathcal{R}p(\mathbf{x})d\mathbf{x}
\end{equation}
と書ける。\\
領域
$\mathcal{R}$
のデータ点の総数$K$は、二項分布に従う。
\begin{equation}
  Bin(K | N, P) = \binom{N}{K} p^K (1 - p)^{N-K}
\end{equation}
よって、$\frac{K}{N}$の分布の平均と分散は
\begin{equation}
  \mathbf{E}[K/N] = P
\end{equation}
\begin{equation}
  var[K/N] = \frac{P(1 - P)}{N}
\end{equation}
となる事がわかる。

$N  \to \infty$ のとき、$\frac{K}{N}$の分布は平均周りで尖った形になり、
\begin{equation}
  K \simeq NP
\end{equation}
となる。\\
また、
$\mathcal{R}$
が小さく、確率密度$p(\mathbf{x})$がこの領域内で一定とみなせるとしたら、
\begin{equation}
  P\simeq p(\mathbf{x})V
\end{equation}
となる。ただし、$V$は、$\mathcal{R}$の体積である。

$V$を固定し、$K$をデータから推定できたとすると、
\begin{equation}
  p(\mathbf{x}) \simeq \frac{K}{NV}
\end{equation}
により、$p(\mathbf{x})$を推定できる。あとは$K$をデータから推定をする部分を考えれば良い。

確率密度を求めたい点を$\mathbf{x}$、この点を中心とする超立方体を$\mathcal{R}$とする。また、観測値の集合を$\{\mathbf{\mathbf{x}_n}\}$とする。
さらに、カーネル関数

  \begin{eqnarray}
    k(\mathbf{u})=\left\{  \begin{array}{ll}
    1 & (\|u_i\| \leq \frac{1}{2}) \\
    0 & (o.w.) \\
    \end{array} \right.
  \end{eqnarray}
を定義する。
この方法では、$k(\frac{\mathbf{x} - \mathbf{x_n}}{h})$は$\mathbf{\mathbf{x}}$を中心とする一辺が$h$の立方体$\mathcal{R}$の内部にデータ点があれば1に、そうでなければ0になる。

立方体$\mathcal{R}$内のデータ点の総数は
\begin{equation}
  K = \sum_{n=1}^{N} k(\frac{\mathbf{x} - \mathbf{x_n}}{h})
\end{equation}
と書け、$\mathbf{\mathbf{x}}$での推定密度は
\begin{equation}
  p(\mathbf{x}) = \frac{1}{N}\sum_{n=1}^{N} \frac{1}{h^D} k(\frac{\mathbf{x} - \mathbf{x_n}}{h})
\end{equation}
となる。上式の確率密度のクラスは、カーネル密度推定法や\textbf{Parzen 推定法}と呼ばれる。

このとき、立方体の縁で、人為的な不連続が生じてしまう。より滑らかなカーネル関数を使えば、より滑らかな密度が得られる。
カーネル関数$\k(\mathbf{u})$には、以下の条件を満たす任意の関数を選ぶことができる。

\begin{equation}
  k(\mathbf{u}) \geq 0
\end{equation}
\begin{equation}
  \int k(\mathbf{u}) d\mathbf{u} = 1
\end{equation}

一般によく選ばれるカーネル関数はガウスカーネルで、次の確率密度モデルになる。

\begin{equation}
  p(\mathbf{x}) = \frac{1}{(2\pi h^2)^{D/2}} exp\{-\frac{(\|\mathbf{\mathbf{x}} - \mathbf{x_n\|)^2}}{2h^2}\}
\end{equation}

ただし、$h$はガウス分布の標準偏差を表す。パラメータ$h$は平滑化パラメータの役割を果たし、$h$が小さすぎると、結果はノイズの多い密度モデルになるが、大きすぎると、データの局所的な性質が捉えにくくなる。最良の密度は、中間的な$h$の値で得られる。
\end{document}
